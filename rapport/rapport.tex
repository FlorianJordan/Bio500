\documentclass[9pt,twocolumn,twoside,]{pnas-new}

% Use the lineno option to display guide line numbers if required.
% Note that the use of elements such as single-column equations
% may affect the guide line number alignment.


\usepackage[T1]{fontenc}
\usepackage[utf8]{inputenc}

% tightlist command for lists without linebreak
\providecommand{\tightlist}{%
  \setlength{\itemsep}{0pt}\setlength{\parskip}{0pt}}


% Pandoc citation processing
\newlength{\cslhangindent}
\setlength{\cslhangindent}{1.5em}
\newlength{\csllabelwidth}
\setlength{\csllabelwidth}{3em}
\newlength{\cslentryspacingunit} % times entry-spacing
\setlength{\cslentryspacingunit}{\parskip}
% for Pandoc 2.8 to 2.10.1
\newenvironment{cslreferences}%
  {}%
  {\par}
% For Pandoc 2.11+
\newenvironment{CSLReferences}[2] % #1 hanging-ident, #2 entry spacing
 {% don't indent paragraphs
  \setlength{\parindent}{0pt}
  % turn on hanging indent if param 1 is 1
  \ifodd #1
  \let\oldpar\par
  \def\par{\hangindent=\cslhangindent\oldpar}
  \fi
  % set entry spacing
  \setlength{\parskip}{#2\cslentryspacingunit}
 }%
 {}
\usepackage{calc}
\newcommand{\CSLBlock}[1]{#1\hfill\break}
\newcommand{\CSLLeftMargin}[1]{\parbox[t]{\csllabelwidth}{#1}}
\newcommand{\CSLRightInline}[1]{\parbox[t]{\linewidth - \csllabelwidth}{#1}\break}
\newcommand{\CSLIndent}[1]{\hspace{\cslhangindent}#1}


\templatetype{pnasresearcharticle}  % Choose template

\title{Rapport}

\author[a]{Marguerite Duchesne}
\author[a]{Florian Jordan}
\author[a]{Anthony St-Pierre}
\author[a]{Simon Grégoire}
\author[a]{Francis Lessard}

  \affil[a]{Université de Sherbrooke, Départment de biologie, 2500
Boulevard de l'Université, Sherbrooke, Québec, J1K 2R1}


% Please give the surname of the lead author for the running footer
\leadauthor{}

% Please add here a significance statement to explain the relevance of your work
\significancestatement{}


\authorcontributions{}



\correspondingauthor{\textsuperscript{} }

% Keywords are not mandatory, but authors are strongly encouraged to provide them. If provided, please include two to five keywords, separated by the pipe symbol, e.g:
 \keywords{  Travaux d'équipe |  collaborations |  Université de
sherbrooke |  Optionel |  Optionel  } 

\begin{abstract}
Nous avons voulu nous intérésser aux collaborations des élèves de
l'Université de Sherbrooke lors de travaux d'équipe pendant leur
parcours dans le baccalauréat en biologie.
\end{abstract}

\dates{This manuscript was compiled on \today}
\doi{\url{www.pnas.org/cgi/doi/10.1073/pnas.XXXXXXXXXX}}

\begin{document}

% Optional adjustment to line up main text (after abstract) of first page with line numbers, when using both lineno and twocolumn options.
% You should only change this length when you've finalised the article contents.
\verticaladjustment{-2pt}



\maketitle
\thispagestyle{firststyle}
\ifthenelse{\boolean{shortarticle}}{\ifthenelse{\boolean{singlecolumn}}{\abscontentformatted}{\abscontent}}{}

% If your first paragraph (i.e. with the \dropcap) contains a list environment (quote, quotation, theorem, definition, enumerate, itemize...), the line after the list may have some extra indentation. If this is the case, add \parshape=0 to the end of the list environment.

\acknow{}

\hypertarget{introduction}{%
\subsection{Introduction}\label{introduction}}

On entend souvent l'expression « ah que le monde est petit ! » lorsque
deux personnes se retrouvent à avoir une connexion qu'on ne suspectait
pas. Certaines études se sont intéressées à ce principe que par un lien
relativement proche, tout le monde se connait à un certain niveau.
Milgram (1967) s'est penché sur le sujet et à testé cette hypothèse que
deux personnes pigées au hasard vont avoir un lien quelconque entre eux
(1). Ce principe peut s'appliquer à l'écologie, car d'un point de vu de
l'évolution, toutes les espèces sont reliées par un ancêtre commun et
pour étudier les réseaux trophiques (2). Ce modèle de « petit monde »
peut donc s'appliquer à grande et petite échelle. Nous avons voulu
tester cette théorie à très petite échelle dans le baccalauréat de la
59e cohorte d'écologie de l'Université de Sherbrooke. Nous nous sommes
donc posés la question si le réseau de collaborations entre les
étudiants du bacc en écologie a les mêmes propriétés que les réseaux
écologiques. Plus spécifiquement, nous avons étudiés si les élèves en
tendances à conserver les mêmes collaborateurs dans tous les travaux ou
s'ils avaient plus tendance à diversifier leurs partenaires. En effet,
il est intéressant de voir si les étudiants ont plusieurs groupe d'amis
ou si au cours du baccalauréat, ils sont restés toujours avec les mêmes
personnes. Nous avons aussi vérifier si le cours de méthode méthode
analytique en biologie (TSB303) a eu beaucoup d'effet dans le réseaux de
collaboration, puisque dans ce cours, les travaux étaient en équipe de
15. On peut donc s'imaginer qu'à lui seul, ce cours ajoute beaucoup de
collaborations entre les étudiants. Pour aider à visulaliser le tout, le
première figure va détailler toutes les colaborations entre tous les
individus de la cohorte, puis plus spécifiquement ne figure qui démontre
uniquements les liens de plus de X collaborations et ensuite cette même
figure, mais en excluant le cours TSB303.

\hypertarget{muxe9thode}{%
\subsection{Méthode}\label{muxe9thode}}

La classe de BIO500 de la session d'hiver 2022 s'est divisé en 9 (à
valider) équipes. Chaque élève de ses équipes a copilé l'ensemble des
cours réaliser lors de leur baccalauréat ainsi que les informations
considérées pertinentes reliées à ces cours dans une première table
commune à l'équipe. Ils ont également copilé dans une seconde table le
nom de chaque coéquipier, l'année de début de leur baccalauréat, le nom
de leur programme ainsi que les informations considérées pertinentes
reliées à chaque individu de l'équipe. Ils ont terminé la copilation des
données par une troisième table. Au sein de cette dernière table, se
trouve l'ensemble des collaborations, c'est-à-dire l'ensemble des noms
avec qui chaque élève a réalisé des travaux d'équipe jusqu'à présent
lors de leur baccalauréat.

Une fois la copilation des données réalisée par chaque équipe, celle-ci
fut partagée et mise en commun. Maintenant indépendentes, les équipes
avaient alors la tâche de fusionner l'ensemble des données ensemble afin
de n'avoir que trois tables contenant l'ensemble des données de la
classe. Au préalable, chaque équipe a dû standardiser les données de
l'ensemble des équipes afin d'obtenir une unité structurelle au sein des
différentes tables. Ces données ont été ensuite intégré dans le système
de gestion de données SQLite3. Afin de répondre à la question posé, les
données d'intérêts ont été extraites via des requêtes et finalement
analysées.

\hypertarget{discussion}{%
\subsection{Discussion}\label{discussion}}

\hypertarget{ruxe9sultats}{%
\subsubsection{Résultats}\label{ruxe9sultats}}

Fig. 1 : Réseau de collaborations des élèves de la 59e cohorte

Fig. 2 : Nombres de collaboration différentes par élève

Fig. 3 : Nombres de collaboration différentes par élèves sans TSB303

\textbf{Voici comment inclure une figure .pdf préalablement générée et
la citer/référencer dans le texte, via son \texttt{label}: Figure
\ref{fig:plot1}}.

\hypertarget{conclusion}{%
\subsection{Conclusion}\label{conclusion}}

\newpage

\hypertarget{bibliographie}{%
\section*{Bibliographie}\label{bibliographie}}
\addcontentsline{toc}{section}{Bibliographie}

\hypertarget{refs}{}
\begin{CSLReferences}{0}{0}
\leavevmode\vadjust pre{\hypertarget{ref-milgram1967small}{}}%
\CSLLeftMargin{1. }
\CSLRightInline{Milgram S (1967) The small world problem.
\emph{Psychology today} 2(1):60--67.}

\leavevmode\vadjust pre{\hypertarget{ref-montoya2002small}{}}%
\CSLLeftMargin{2. }
\CSLRightInline{Montoya JM, Solé RV (2002)
\href{https://doi.org/10.1006/jtbi.2001.2460}{Small {World} {Patterns}
in {Food} {Webs}}. \emph{Journal of Theoretical Biology}
214(3):405--412.}

\end{CSLReferences}



% Bibliography
% \bibliography{pnas-sample}

\end{document}
