\documentclass[9pt,twocolumn,twoside,]{pnas-new}

% Use the lineno option to display guide line numbers if required.
% Note that the use of elements such as single-column equations
% may affect the guide line number alignment.


\usepackage[T1]{fontenc}
\usepackage[utf8]{inputenc}

% tightlist command for lists without linebreak
\providecommand{\tightlist}{%
  \setlength{\itemsep}{0pt}\setlength{\parskip}{0pt}}




\templatetype{pnasresearcharticle}  % Choose template

\title{Rapport}

\author[a]{Marguerite Duchesne}
\author[a]{Florian Jordan}
\author[a]{Anthony St-Pierre}
\author[a]{Simon Grégoire}
\author[a]{Francis Lessard}

  \affil[a]{Université de Sherbrooke, Départment de biologie, 2500 Boulevard de
l'Université, Sherbrooke, Québec, J1K 2R1}


% Please give the surname of the lead author for the running footer
\leadauthor{}

% Please add here a significance statement to explain the relevance of your work
\significancestatement{}


\authorcontributions{}



\correspondingauthor{\textsuperscript{} }

% Keywords are not mandatory, but authors are strongly encouraged to provide them. If provided, please include two to five keywords, separated by the pipe symbol, e.g:
 \keywords{  Travaux d'équipe |  collaborations |  Université de sherbrooke |  Optionel |  Optionel  } 

\begin{abstract}
Nous avons voulu nous intérésser aux collaborations des élèves de
l'Université de Sherbrooke lors de travaux d'équipe pendant leur
parcours dans le baccalauréat en biologie.
\end{abstract}

\dates{This manuscript was compiled on \today}
\doi{\url{www.pnas.org/cgi/doi/10.1073/pnas.XXXXXXXXXX}}

\begin{document}

% Optional adjustment to line up main text (after abstract) of first page with line numbers, when using both lineno and twocolumn options.
% You should only change this length when you've finalised the article contents.
\verticaladjustment{-2pt}



\maketitle
\thispagestyle{firststyle}
\ifthenelse{\boolean{shortarticle}}{\ifthenelse{\boolean{singlecolumn}}{\abscontentformatted}{\abscontent}}{}

% If your first paragraph (i.e. with the \dropcap) contains a list environment (quote, quotation, theorem, definition, enumerate, itemize...), the line after the list may have some extra indentation. If this is the case, add \parshape=0 to the end of the list environment.

\acknow{}

\hypertarget{introduction}{%
\section{Introduction}\label{introduction}}

\textbf{Voici un exemple de citation (1), et en voici deux autres
(1--3).}

On entend souvent l'expression « ah que le monde est petit ! » lorsque
deux personnes se retrouvent à avoir une connexion qu'on ne suspectait
pas. Certaines études se sont intéressées à ce principe que par un lien
relativement proche, tout le monde se connait. Milgram (1967) s'est
penché sur le sujet et à testé cette hypothèse que deux personnes pigées
au hasard vont avoir un lien quelconque entre eux {[}@. Ce principe peut
s'appliquer à l'écologie, car d'un point de vu de l'évolution, toutes
les espèces sont reliées par un ancêtre commun. Ce modèle de « petit
monde » peut donc s'appliquer à grande et petite échelle. Nous avons
voulu tester cette théorie à petite échelle. Nous nous sommes donc posés
la question si le réseau de collaborations entre les étudiants du bacc
en écologie a les mêmes propriétés que les réseaux écologiques. Plus
spécifiquement, nous avons étudiés si les élèves en tendances à
conserver les mêmes collaborateurs dans tous les travaux ou s'ils
avaient plus tendance à diversifier leurs partenaires.

\hypertarget{muxe9thode}{%
\section{Méthode}\label{muxe9thode}}

Per recepto pugnes. Soror est adiit nusquam, in arserunt nondum tempore
bracchia. Pater tempora limen. Falcata parentibus dolor et vobis aranea!
Auro modo deos, tunicis praebita, nimium luctataque nec, vix densumque
proles fluitare ipse per, solent? Pondera abrumpit cum quaeque iuris.
Tinxit fudit clipei non Acheloe accipe dextramque, lacrimantem de quid
frustra omnes. Indigenae exta quamque modo meae, detur idem curas prope
erat. Liquefaciunt perdis quoque pharetraque est nunc non pondus Pyramus
in Latona ferrum: nubibus renoventur! Fallacia sonuit Proreus, aurora:
omen cur a moenia ore ego narret, ego. Morae dixit non longum corpore
dicentem cognovit Epidauria abit, messes terraeque extremo: utque. Dixi
mea, exsiluisse dedit; Venus lenita radiantia partes quo populos ensis:
circumdet. Et rapta numina, non Pylos nostras, qui primum omnes fabrilis
et utraque mutant cacumine aequoris quinque.

\hypertarget{premiuxe8re-partie}{%
\subsection{Première partie}\label{premiuxe8re-partie}}

Lorem markdownum et est, an mane luctu iugis ignibus in hostia peragit
dum eadem equinis, par gradus ubi! Silet canities sine dedecet, fides
virum capro: loquuntur ante. Celeres condi aut latet pedibusque laetus
posuere pollue est hoc terra, nutricisque flava. Eurytus auris. Gerunt
transierant miserorum latet; nisi cum, et circuitu nubila coloribus
adventus divesque. Loca partibus breve et unum maior stellis inpia et
luporum. Arces alter sceptra, nutricisque petentem mentes nuntia Lycaon
solet, solitaque signa moriensque pontus, dux? Corpora testataque, novo
tumidum. Eurytus auris. Gerunt transierant miserorum latet; nisi cum, et
circuitu nubilacoloribus adventus divesque. Loca partibus breve et unum
maior stellis inpia etluporum. Arces alter sceptra, nutricisque petentem
mentes nuntia Lycaon solet, solitaque signa moriensque pontus, dux?
Corpora testataque, novo tumidum.

\hypertarget{deuxiuxe8me-partie}{%
\subsection{Deuxième partie}\label{deuxiuxe8me-partie}}

Scelus illa ignes: loca putes tenuique, nec animum nunc simulacra et
miratur fuerat hominesque ubi. Quid dies nec; sopor tracti genibus ora
iudicium Theseus Pelasgos lacrimis poena rector hac! Nitidum dextro.
Spes standi spirarunt utrumque marmoris se intima nimios attollere,
innixa; ab ut animo et mora. Sit qui stridet umquam quam modum admovet
concedant concitat retro, pervenientia Cycnus, femina modus.

\hypertarget{ruxe9sultats}{%
\section{Résultats}\label{ruxe9sultats}}

Fig. 1 : Réseau de collaborations

Fig. 2 : Nombres de collaboration par élève

Fig. 3 : Nombres de collaboration avec chaque partenaire

\textbf{Voici comment inclure une figure .pdf préalablement générée et
la citer/référencer dans le texte, via son \texttt{label}: Figure
\ref{fig:plot1}}.

\hypertarget{discussion}{%
\section{Discussion}\label{discussion}}

Per recepto pugnes. Soror est adiit nusquam, in arserunt nondum tempore
bracchia. Pater tempora limen. Falcata parentibus dolor et vobis aranea!
Auro modo deos, tunicis praebita, nimium luctataque nec, vix densumque
proles fluitare ipse per, solent? Pondera abrumpit cum quaeque iuris.
Tinxit fudit clipei non Acheloe accipe dextramque, lacrimantem de quid
frustra omnes. Indigenae exta quamque modo meae, detur idem curas prope
erat. Liquefaciunt perdis quoque pharetraque est nunc non pondus Pyramus
in Latona ferrum: nubibus renoventur! Fallacia sonuit Proreus, aurora:
omen cur a moenia ore ego narret, ego. Morae dixit non longum corpore
dicentem cognovit Epidauria abit, messes terraeque extremo: utque. Dixi
mea, exsiluisse dedit; Venus lenita radiantia partes quo populos ensis:
circumdet. Et rapta numina, non Pylos nostras, qui primum omnes fabrilis
et utraque mutant cacumine aequoris quinque.

\hypertarget{conclusion}{%
\section{Conclusion}\label{conclusion}}

Ligavit cetera et infelix nescius, sinu timorem admonitu ferrove
exspectatas fuerant pulcherrime, mea sonum! Formae munera, est trahendo
regem supplice. Verba lacrimis gradus, verbis iam tempore aratri sed, at
imagine momordit. Teque esse, ante pariter desere in quae: illa socium
inque meae: et ipso canos: peteretis? Temptanda Daedalus et palmis
incumbere tempora ambit.

\newpage

\hypertarget{bibliographie}{%
\section*{Bibliographie}\label{bibliographie}}
\addcontentsline{toc}{section}{Bibliographie}

\hypertarget{refs}{}
\leavevmode\hypertarget{ref-berard1994embedding}{}%
1. Bérard P, Besson G, Gallot S (1994) Embedding riemannian manifolds by
their heat kernel. \emph{Geometric \& Functional Analysis GAFA}
4(4):373--398.

\leavevmode\hypertarget{ref-belkin2002using}{}%
2. Belkin M, Niyogi P (2002) Using manifold stucture for partially
labeled classification. \emph{Advances in Neural Information Processing
Systems}, pp 929--936.

\leavevmode\hypertarget{ref-coifman2005geometric}{}%
3. Coifman RR, et al. (2005) Geometric diffusions as a tool for harmonic
analysis and structure definition of data: Diffusion maps.
\emph{Proceedings of the National Academy of Sciences of the United
States of America} 102(21):7426--7431.



% Bibliography
% \bibliography{pnas-sample}

\end{document}

